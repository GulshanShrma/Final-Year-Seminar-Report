\pagebreak 
\section*{3. Architecture}
\addcontentsline{toc}{chapter}{Architecture}

Decentralized cryptocurrency is produced by the entire cryptocurrency system collectively, at a rate which is defined when the system is created and which is publicly known. In centralized banking and economic systems such as the Federal Reserve System, corporate boards or governments control the supply of currency by printing units of fiat money or demanding additions to digital banking ledgers. In the case of decentralized cryptocurrency, companies or governments cannot produce new units, and have not so far provided backing for other firms, banks or corporate entities which hold asset value measured in it. The underlying technical system upon which decentralized cryptocurrencies are based was created by the group or individual known as Satoshi Nakamoto.\vspace{.3cm}

As of May 2018, over 1,800 cryptocurrency specifications existed. Within a cryptocurrency system, the safety, integrity and balance of ledgers is maintained by a community of mutually distrustful parties referred to as miners: who use their computers to help validate and timestamp transactions, adding them to the ledger in accordance with a particular timestamping scheme.\vspace{.3cm}

Most cryptocurrencies are designed to gradually decrease the production of that currency, placing a cap on the total amount of that currency that will ever be in circulation. Compared with ordinary currencies held by financial institutions or kept as cash on hand, cryptocurrencies can be more difficult for seizure by law enforcement.\vspace{.3cm}

\subsection*{3.1 Blockchain}
The validity of each cryptocurrency's coins is provided by a blockchain. A blockchain is a continuously growing list of records, called blocks, which are linked and secured using cryptography. Each block typically contains a hash pointer as a link to a previous block, a timestamp and transaction data. By design, blockchains are inherently resistant to modification of the data. It is "an open, distributed ledger that can record transactions between two parties efficiently and in a verifiable and permanent way". For use as a distributed ledger, a blockchain is typically managed by a peer-to-peer network collectively adhering to a protocol for validating new blocks. Once recorded, the data in any given block cannot be altered retroactively without the alteration of all subsequent blocks, which requires collusion of the network majority.\vspace{.3cm}

Blockchains are secure by design and are an example of a distributed computing system with high Byzantine fault tolerance. Decentralized consensus has therefore been achieved with a blockchain.\vspace{.3cm}

\subsection*{3.2 Nodes}
In the world of Cryptocurrency, a node is a computer that connects to a cryptocurrency network. The node supports the relevant cryptocurrency's network through either; relaying transactions, validation or hosting a copy of the blockchain. In terms of relaying transactions each network computer (node) has a copy of the blockchain of the cryptocurrency it supports, when a transaction is made the node creating the transaction broadcasts details of the transaction using encryption to other nodes throughout the node network so that the transaction (and every other transaction) is known.\vspace{.3cm}

Node owners are either volunteers, those hosted by the organisation or body responsible for developing the cryptocurrency blockchain network technology or those that are enticed to host a node to receive rewards from hosting the node network.[40]\vspace{.3cm}

\subsection*{3.3 Timestamping}
Cryptocurrencies use various timestamping schemes to "prove" the validity of transactions added to the blockchain ledger without the need for a trusted third party.\vspace{.3cm}

The first timestamping scheme invented was the proof-of-work scheme. The most widely used proof-of-work schemes are based on SHA-256 and scrypt.\vspace{.3cm}

Some other hashing algorithms that are used for proof-of-work include CryptoNight, Blake, SHA-3, and X11.\vspace{.3cm}

The proof-of-stake is a method of securing a cryptocurrency network and achieving distributed consensus through requesting users to show ownership of a certain amount of currency. It is different from proof-of-work systems that run difficult hashing algorithms to validate electronic transactions. The scheme is largely dependent on the coin, and there's currently no standard form of it. Some cryptocurrencies use a combined proof-of-work and proof-of-stake scheme.
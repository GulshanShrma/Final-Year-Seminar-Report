\chapter*{1. Introduction}
\addcontentsline{toc}{chapter}{Introduction}
	 Cryptocurrencies, or virtual currencies, are digital means of exchange that uses cryptography for security. The word ‘crypto’ comes from the ancient greek word, ‘kryptós’, which means hidden or private. A digital currency that is created and used by private individuals or groups has multiple benefits.\vspace{.3cm}
	 
	 \subsection*{1.1 What Is a Cryptocurrency?}
	 
	 Cryptocurrencies challenge the orthodoxy of how a currency works in ways that excite some and worry others. So, what exactly is cryptocurrency and why is it different? Unlike other currencies, all cryptocurrencies are entirely digital. No cryptocurrency prints money or mints coins. Everything is done online. Conventional forms of currency are generated by government and then circulated in the economy, via banks.\vspace{.3cm}
	 
	 \subsection*{1.2 Value of a Cryptocurrency}
	 
	 Cryptocurrencies do not rely on either of these institutions. Instead, cryptocurrency is decentralized. In other words, it is created, exchanged and regulated by its users. Cryptocurrencies are digitally mined. Mining precious metals has been used as a means of giving value to money. The question of how cryptocurrencies have value is complex, and reveals that any currency derives its worth from faith in its purchasing power. All currencies require a system that guards against misuse and fraud.\vspace{.3cm}
	 
	 \subsection*{1.3 Blockchain}
	 
	 In banking, this is done with ledgers which track the flow of money through accounts. With cryptocurrency, the task is undertaken with blockchain using a form of maths called cryptology. Blockchain is a secure record of every single transaction made using a cryptocurrency. Verified transactions are added to the blockchain as part of the mining process. Mining is therefore not just about creating new money but also validating transactions. While it’s possible to buy cryptocurrency- all you need is a digital wallet as part of a free app or a cryptocurrency tax software — finding places that will accept it, the variable transaction charges and volatile exchange rates make buying and selling with it difficult.\vspace{.3cm}

	 \subsection*{1.4 Applications of Cryptocurrency}
	 
	 Cryptocurrency could transform the way we do transactions. The so-called distributed ledger technology behind blockchain can be integrated into all sorts of business processes that require trust among multiple parties. That’s because blockchains store information that are both secure and transparent. Pretty exciting, but how is that possible? For one thing, because of the blocks themselves. Now, rather than a long string of records, information in a blockchain is cut up into sealed blocks. Thanks to the use of cryptography, it is impossible to change or counterfeit the records in the block. But what’s inside these blocks?\vspace{.3cm}
	 
	 Each block contains certain data, for example when selling an exclusive painting you want the block to have information on the name of the painting, the artist the previous owner, the new owner, the time of the sale and transaction. Next to the data, each block has an identifiable hash. This is a unique code, that functions like a fingerprint.\vspace{.3cm}
	 
	 \section*{1.5 Advantages and Disadvantages of Cryptocurrency}
	 \subsection*{1.5.1 Advantages}
	 Cryptocurrencies hold the promise of making it easier to transfer funds directly between two parties, without the need for a trusted third party like a bank or credit card company. These transfers are instead secured by the use of public keys and private keys and different forms of incentive systems, like Proof of Work or Proof of Stake.
	 
	 In modern cryptocurrency systems, a user's "wallet," or account address, has a public key, while the private key is known only to the owner and is used to sign transactions. Fund transfers are completed with minimal processing fees, allowing users to avoid the steep fees charged by banks and financial institutions for wire transfers.
	 
	 \subsection*{1.5.2 Disadvantages} 
	 The semi-anonymous nature of cryptocurrency transactions makes them well-suited for a host of illegal activities, such as money laundering and tax evasion. However, cryptocurrency advocates often highly value their anonymity, citing benefits of privacy like protection for whistleblowers or activists living under repressive governments. Some cryptocurrencies are more private than others. 
	 
	 Bitcoin, for instance, is a relatively poor choice for conducting illegal business online, since the forensic analysis of the Bitcoin blockchain has helped authorities arrest and prosecute criminals. More privacy-oriented coins do exist, however, such as Dash, Monero, or ZCash, which are far more difficult to trace.
	
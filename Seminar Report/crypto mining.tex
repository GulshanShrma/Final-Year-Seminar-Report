\section*{4. Crypto Mining}
\addcontentsline{toc}{chapter}{Crypto Mining}

In cryptocurrency networks, mining is a validation of transactions. For this effort, successful miners obtain new cryptocurrency as a reward. The reward decreases transaction fees by creating a complementary incentive to contribute to the processing power of the network. The rate of generating hashes, which validate any transaction, has been increased by the use of specialized machines such as FPGAs and ASICs running complex hashing algorithms like SHA-256 and scrypt. This arms race for cheaper-yet-efficient machines has existed since the day the first cryptocurrency, bitcoin, was introduced in 2009. With more people venturing into the world of virtual currency, generating hashes for this validation has become far more complex over the years, with miners having to invest large sums of money on employing multiple high performance ASICs. Thus the value of the currency obtained for finding a hash often does not justify the amount of money spent on setting up the machines, the cooling facilities to overcome the heat they produce, and the electricity required to run them. Favorite regions for mining are those with cheap electricity or a cold climate. As of July 2019, bitcoin's electricity consumption is estimated to about 7 gigawatts, 0.2\% of the global total, or equivalent to that of Switzerland.

Some miners pool resources, sharing their processing power over a network to split the reward equally, according to the amount of work they contributed to the probability of finding a block. A "share" is awarded to members of the mining pool who present a valid partial proof-of-work.

As of February 2018, the Chinese Government halted trading of virtual currency, banned initial coin offerings and shut down mining. Some Chinese miners have since relocated to Canada. One company is operating data centers for mining operations at Canadian oil and gas field sites, due to low gas prices. In June 2018, Hydro Quebec proposed to the provincial government to allocate 500 MW to crypto companies for mining. According to a February 2018 report from Fortune, Iceland has become a haven for cryptocurrency miners in part because of its cheap electricity.

In March 2018, the city of Plattsburgh in upstate New York put an 18-month moratorium on all cryptocurrency mining in an effort to preserve natural resources and the "character and direction" of the city.

\subsection*{4.1 GPU price rise}
An increase in cryptocurrency mining increased the demand for graphics cards (GPU) in 2017. (The computing power of GPUs makes them well-suited to generating hashes.) Popular favorites of cryptocurrency miners such as Nvidia's GTX 1060 and GTX 1070 graphics cards, as well as AMD's RX 570 and RX 580 GPUs, doubled or tripled in price – or were out of stock.[50] A GTX 1070 Ti which was released at a price of \$450 sold for as much as \$1100. Another popular card GTX 1060's 6 GB model was released at an MSRP of \$250, sold for almost \$500. RX 570 and RX 580 cards from AMD were out of stock for almost a year. Miners regularly buy up the entire stock of new GPU's as soon as they are available.

Nvidia has asked retailers to do what they can when it comes to selling GPUs to gamers instead of miners. "Gamers come first for Nvidia," said Boris Böhles, PR manager for Nvidia in the German region.

\subsection*{4.2 Wallets}

An example paper printable bitcoin wallet consisting of one bitcoin address for receiving and the corresponding private key for spending
A cryptocurrency wallet stores the public and private "keys" (address) or seed which can be used to receive or spend the cryptocurrency. With the private key, it is possible to write in the public ledger, effectively spending the associated cryptocurrency. With the public key, it is possible for others to send currency to the wallet.

There exist multiple methods of storing keys or seed in a wallet from using paper wallets which are traditional public, private or seed keys written on paper to using hardware wallets which are dedicated hardware to securely store your wallet information, using a digital wallet which is a computer with a software hosting your wallet information, hosting your wallet using an exchange where cryptocurrency is traded. or by storing your wallet information on a digital medium such as plaintext.

\subsection*{4.3 Anonymity}
Bitcoin is pseudonymous rather than anonymous in that the cryptocurrency within a wallet is not tied to people, but rather to one or more specific keys (or "addresses"). Thereby, bitcoin owners are not identifiable, but all transactions are publicly available in the blockchain. Still, cryptocurrency exchanges are often required by law to collect the personal information of their users.[citation needed]

Additions such as Monero, Zerocoin, Zerocash and CryptoNote have been suggested, which would allow for additional anonymity and fungibility.